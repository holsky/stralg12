\documentclass[a4paper,10pt]{article}

\usepackage{ucs}
\usepackage[utf8x]{inputenc}
\usepackage[english]{babel}
\usepackage{fontenc}
\usepackage{graphicx}
\usepackage[a4paper, top=2cm]{geometry}
\usepackage{amssymb}

\usepackage[dvips]{hyperref}
\title{String Algorithms 2013 - Assignment 3}
\author{Holger Schmeisky  201211113, Matus Tomlein 201210962}
\date{29.05.2013}

\begin{document}

\maketitle

\subsection*{State of the algorithms}
The algorithms work correctly for all inputs we tested.

\subsection*{Insights}


\subsection*{Problems}


\subsection*{Evaluation of correctness}

We created a script that evaluates the results of our KMP and BA algorithm
implementations and compares them against the results of a simple algorithm
that searches for a given pattern by comparing substrings at each index
in the given string.
We used a long DNA string as an input and generated random search patterns
based on the input string.

According to the evaluation, both our KMP and BA implementations worked
correctly.

\clearpage
\subsection*{Evaluation}

\subsection*{Comparison with suffix tree}

The question was:
\emph{How many times should you search for a pattern in the same text before it faster to use search from mandatory project 1 (based on constructing the suffix tree) rather than search-ba or search-kmp?}

As soon as searching for the pattern K times in BA and KMP takes longer
than constructing the suffix tree for the input, we can say that searching in
the suffix tree is more efficient.
If we store the suffix tree, we have to pay the cost of creating it only once,
afterwards searching for any pattern takes just $\mathcal{O}(M)$.
However, in KMP and BA it always takes $\mathcal{O}(M+N)$ to search for a pattern.

\end{document}
